\section{Instantiation from the q-SDH Assumption}
In the following let $\pms \coloneqq (p,\G) $ be a set of system parameters, where 
$\G $ is a cyclic group of prime order $p$.

\begin{definition}[$q$-SDH Assumption]
We say an adversary $\adv$ breaks the $q$-strong Diffie Hellman ($q$-SDH)assumption if it's running time is bounded by $t$ and
$$\Pr[(s,g^{\frac{1}{s+x}}) \sample \adv(g,g^x,...,g^{x^q})] \geq \epsilon,$$
where 
$g \sample \G$ and $x \sample \mathbb{Z}_p^*$.
We require that the $q$-SDH assumption holds meaning that no adversary can $(t, \epsilon)$ break the $q$-SDH problem for a polynomial $t$ and a non-negligible $\epsilon$.
\end{definition}

\begin{figure}[htb!]
    \centering
    \nicoresetlinenr
    \fbox{
        \begin{minipage}[t]{0.5\textwidth}
            \underline{$\IDGen(\pms)$:}
            \begin{nicodemus}
            	\item $g \sample \G$
		\item $x \sample \mathbb{Z}_p$
                \item $\sk \coloneqq (g,x)$
                \item $X = g^x$
                \item $\pk \coloneqq (g,X)$
                \item $\ChSet \coloneqq \mathbb{Z}_p$
                \item $\pcreturn (\sk,\pk)$
            \end{nicodemus}
            \underline{$\IDV(\pk,\tau,\com,\c,\s)$}:
            \begin{nicodemus}
            	\item $\pcparse \com = (\hat{g},R,\hat{R})$
      		\item \pcif $R = g^\s \cdot(X\cdot g^{\tau})^{-\c} \land$
		\item$ \quad \hat{R} = \hat{g}^\s \cdot (g\cdot \hat{g}^{-\tau} )^\c$
                 \item \quad \pcthen \pcreturn 1
                 \item \pcelse \pcreturn 0
            \end{nicodemus}
            

            
        \end{minipage}
        \begin{minipage}[t]{0.5\textwidth}
            \underline{$\IDP_1(\sk,\tau):$}
            \begin{nicodemus}
            	\item $r \sample \mathbb{Z}_p$
                \item $\st \coloneqq (\tau,r )$
                \item $\hat{g} \coloneqq g^{\frac{1}{x+\tau}}$
                \item $R \coloneqq g^r$
                \item $\hat{R} \coloneqq \hat{g}^r$
                \item $\com := (\hat{g},R,\hat{R})$
                \item $\pcreturn (\com,\st)$
            \end{nicodemus}
            
              \underline{$\IDP_2(\sk,\st,\c):$}
            \begin{nicodemus}
            	\item $\pcparse \st = (\tau,r)$
		\item $\pcparse \sk = x$
                \item $\pcreturn \s =\c\cdot(x+\tau) + r \mod p$
            \end{nicodemus}
                        \underline{$\IDAltV(\sk,\tau,\com, \c,\s)$}:
            \begin{nicodemus}
                \item \pcparse $\com = (\hat{g},R,\hat{R})$
                \item \pcparse $\sk = x$
                \item \pcif $\hat{g} = g^{\frac{1}{x+\tau}}$ 
                \item \quad \pcthen \pcreturn 1
                \item \pcelse \pcreturn 0
            \end{nicodemus}
        \end{minipage}
    }
    \caption{Instantiation 1}
    \label{fig:prf-security}
\end{figure}

We describe the identification scheme  $\ID_{q-\text{SDH}} \coloneqq(\IDGen, \IDP=(\IDP_1,\IDP_2) ,\ChSet, \IDV)$ and it's alternative verification $\IDAltV$ as depicted in figure 4.

\begin{theorem}
Suppose that there exists a $(t, q, \epsilon)$-forger $\fdv$ breaking the $\imp^{\IDAltV}$ of the $\ID_{q-\text{SDH}}$ identification scheme. Then there exists an adversary $\adv$ that $(t',q+1,\epsilon')-$ breaks the $q+1$-SDH assumption with $t \approx t'$ and 
$\epsilon' \geq ?.$
\end{theorem}
\begin{proof}
The $q$-SDH adversary $\adv$ receives $d_0,...,d_q$ as inputs where $d_i = g^{x^i}$ and simulates the $q$-SDH experiment as follows
\subsection*{Key Generation:} The adversary $\adv$ first chooses random $\tau_1,...,\tau_q$ from $\mathbb{Z}_p$. Let $f$ be a univariate polynomial defined as 
$f(X) = \prod_{i=1}^q (X+\tau_i)$. Expand $f$ and write $f(X) = \sum_{i=0}^q \alpha_i X^i$ where
$\alpha_0,...,\alpha_q \in \mathbb{Z}_p$ are coefficients of the polynomial $f$. Adversary $\adv$ chooses a random $\theta \in \mathbb{Z}^*_p$, and computes
$$g_1 \leftarrow \prod_{i=0}^q d_i^{\theta\alpha_i}$$
which essentially means $g_1 = g^{\theta f(x)}$. $\adv$ can also calculates 
$X = g_1^x = g^{xf(x)}$ similarly since $Xf(X)$ has a degree equal to $q+1$.
\\
Adversary $\adv$ returns $(g_1,X)$ as the public key to $\fdv$. This is indistinguishable from the normal key generation for $\mathbb{F}$ since $g_1$ is randomly distributed in $\G$ and $X$ is correctly computed.

\subsection*{Transcript Generation:}
Now adversary $\adv$ compute $(\com_i,\c_i,\s_i)$ for $\tau_i$.
\\
$\adv$ computes
$\hat{g}_i = g_1^{\frac{1}{x+\tau_i}}$ for $i = 1,..,q$. To do so, let $f_i$ be defined as
$$f_i(X) = \frac{f_i(X)}{X+\tau_i} = \prod_{j=1, j \neq i}^q (X+ \tau_j).$$
As before, we write $f_i$ as 
$f_i(X) = \sum_{j=0}^{q-1} \beta_jX^j$ while calculating its coefficient.  Now $\adv$ can compute
$$\hat{g}_i = \prod_{j=0}^{q-1} d_j^{\theta\beta_j}$$
hence
$$\hat{g}_i = g^{\theta f_i(X)} =  g_1^{\frac{1}{x+\tau_i}}.$$




 Then $\adv$ chooses  $\c_i,\s_i\sample\mathbb{Z}_p$ and computes
$$R = g_1^\s \cdot(X\cdot g_1^{\tau})^{-\c} $$
$$\hat{R} = \hat{g}^\s \cdot (g\cdot \hat{g}^{-\tau} )^\c.$$
Now $\adv$ returns $(\com_i = (\hat{g},R,\hat{R}),\c_i,\s_i)$ to $\fdv$ and this is indistinguishable from the normal transcript generation for $\fdv$ since if we define $r$ to be $r = \s-\c x$ then $R = g_1^x$ and 
$\hat{R} = \hat{g}^x$ and also since $\s$ and $\c$ are uniformly distributed in $\mathbb{Z}_p$ so is $r$.
\subsection*{Breaking the $q+1$-SDH:}
Eventually forger $\fdv$ returns a forgery $(\tau^*,\com^*,\c^*,\s^*)$ we assume that $\fdv$ wins the game and thus $\tau^* \notin \{\tau_1,...,\tau_q\}$ and 
$\IDAltV(\sk, \tau^*,\com^*,\c^*,\s^*) = 1$ which means if we parse $\com^*$ as $(\hat{g}^*,R^*,\hat{R}^*)$
$$\hat{g} = g_1^{\frac{1}{x+\tau^*}} = g^{\frac{\theta f(x)}{x+\tau^*}}$$
Using long division we can write $f(X)$ as $f(X) = (X+\tau^*)\alpha(X) + \beta$  where the coefficients of $\alpha(X) = \sum_{i=0}^{q-1} \alpha_iX^i$ are easily computable. So we can write $\frac{f(X)}{X+\tau^*}$ as $\alpha(X) + \frac{\beta}{X+\tau^*}$ and 
$$\hat{g} = g^{\theta\cdot(\alpha(X) + \frac{\beta}{X+\tau^*})}.$$
Since $\{\tau_1,...,\tau_q\}$ are the set of roots for $f(X)$ and $\tau^*$ is not in this set $X+\tau^*$ does not divide $f(X)$ and so $\beta \neq 0$. Now adversary $\adv$ can compute
$$w \leftarrow \Big{(} \hat{g}^{\frac{1}{\theta}} \cdot \prod_{i=0}^{q-1} d_i^{-\alpha_i}\Big{)}^{\frac{1}{\beta}}$$
Hence,
$$w =\Big{(} g_1^{\alpha(X)}\cdot g_1^{\frac{\beta}{x+\tau^*}}  \prod_{i=0}^{q-1} d_i^{-\alpha_i}\Big{)}^{\frac{1}{\beta}} = g_1^{\frac{1}{x+\tau^*}}.$$
Adversary $\adv$ returns the pair $(\tau^*, w)$ as the solution to the $q+1-$SDH problem.

\end{proof}  