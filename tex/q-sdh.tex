\section{Instantiation of dual tag ID from q-SDH}
Throughout thus section let $\pms \coloneqq (p,\G) $ be a set of system parameters, where 
$\G $ is a cyclic group of prime order $p$.

\begin{definition}[$q$-SDH Assumption]
We say an adversary $\adv$ breaks the $q$-strong Diffie Hellman ($q$-SDH)assumption if it's running time is bounded by $t$ and
$$\Pr[(s,g^{\frac{1}{s+x}}) \sample \adv(g,g^x,...,g^{x^q})] \geq \epsilon,$$
where 
$g \sample \G$ and $x \sample \mathbb{Z}_p^*$.
\end{definition}

\begin{lemma}
Let $f$ be a polynomial
$$f(X) = \prod_{i =1}^{i = q} (X+\tau_i)$$
for some $\tau_i \in \mathbb{Z}_p$. Given $\{g^{x_i}\}_{i=0,...,q}$, let us define $g_1 = g^{\theta f(x)}$. For any $\tau_i$ where $i \in [1,q]$ it is easy to compute $g_1^{\frac{1}{x+\tau_i}}$ and given $g_1^{\frac{1}{x + \tau}}$ where $\tau \notin \{\tau_1,...,\tau_q\}$, one can easily compute $g^{\frac{1}{x+\tau}}$.
\end{lemma}
\begin{proof}
Reference.
\end{proof}

\begin{figure}[htb!]
    \centering
    \nicoresetlinenr
    \fbox{
        \begin{minipage}[t]{0.5\textwidth}
            \underline{$\IDGen(\pms)$:}
            \begin{nicodemus}
            	\item $g \sample \G$
		\item $x \sample \mathbb{Z}_p$
                \item $\sk \coloneqq (g,x)$
                \item $X = g^x$
                \item $\pk \coloneqq (g,X)$
                \item $\ChSet \coloneqq \mathbb{Z}_p$
                \item $\pcreturn (\sk,\pk)$
            \end{nicodemus}
            \underline{$\IDV(\pk,\tau,\com,\c,\s)$}:
            \begin{nicodemus}
            	\item $\pcparse \com = (\hat{g},R,\hat{R})$
      		\item \pcif $R = g^\s \cdot(X\cdot g^{\tau})^{-\c} \land$
		\item$ \quad \hat{R} = \hat{g}^\s \cdot (g\cdot \hat{g}^{-\tau} )^\c$
                 \item \quad \pcthen \pcreturn 1
                 \item \pcelse \pcreturn 0
            \end{nicodemus}
            

            
        \end{minipage}
        \begin{minipage}[t]{0.5\textwidth}
            \underline{$\IDP_1(\sk,\tau):$}
            \begin{nicodemus}
            	\item $r \sample \mathbb{Z}_p$
                \item $\st \coloneqq (\tau,r )$
                \item $\hat{g} \coloneqq g^{\frac{1}{x+\tau}}$
                \item $R \coloneqq g^r$
                \item $\hat{R} \coloneqq \hat{g}^r$
                \item $\com := (\hat{g},R,\hat{R})$
                \item $\pcreturn (\com,\st)$
            \end{nicodemus}
            
              \underline{$\IDP_2(\sk,\st,\c):$}
            \begin{nicodemus}
            	\item $\pcparse \st = (\tau,r)$
		\item $\pcparse \sk = x$
                \item $\pcreturn \s =\c\cdot(x+\tau) + r \mod p$
            \end{nicodemus}
                        \underline{$\IDAltV(\sk,\tau,\com, \c,\s)$}:
            \begin{nicodemus}
                \item \pcparse $\com = (\hat{g},R,\hat{R})$
                \item \pcparse $\sk = x$
                \item \pcif $\hat{g} = g^{\frac{1}{x+\tau}}$ 
                \item \quad \pcthen \pcreturn 1
                \item \pcelse \pcreturn 0
            \end{nicodemus}
        \end{minipage}
    }
    \caption{Instantiation 1}
    \label{fig:prf-security}
\end{figure}

We describe the identification scheme  $\ID_{q-\text{SDH}} \coloneqq(\IDGen, \IDP=(\IDP_1,\IDP_2) ,\ChSet, \IDV)$ and it's alternative verification $\IDAltV$ as depicted in figure 4.

\begin{theorem}
Suppose that there exists a $(t, q, \epsilon)$-forger $\fdv$ breaking the $\imp^{\IDAltV}$ of the $\ID_{q-\text{SDH}}$ identification scheme. Then there exists an adversary $\adv$ that $(t',q+1,\epsilon')-$ breaks the $q+1$-SDH assumption with $t \approx t'$ and 
$\epsilon' \geq \epsilon.$
\end{theorem}
\begin{proof}
The $q$-SDH adversary $\adv$ receives $d_0,...,d_q$ as inputs where $d_i = g^{x^i}$ and simulates the $q$-SDH experiment as follows
\subsection*{Key Generation:} 
Adversary $\adv$ first chooses random $\tau_1,...,\tau_q$ from $\mathbb{Z}_p$. Now $\adv$ defines $g_1 = g^{\theta f(x)}$ as in Lemma 2. $\adv$ can also calculates 
$X = g_1^x = g^{xf(x)}$ similarly since $Xf(X)$ has a degree equal to $q+1$.
\\
Adversary $\adv$ returns $(g_1,X)$ as the public key to $\fdv$. This is indistinguishable from the normal key generation for $\mathbb{F}$ since $g_1$ is randomly distributed in $\G$ and $X$ is correctly computed.

\subsection*{Transcript Generation:}
Now adversary $\adv$ must compute $(\com_i,\c_i,\s_i)$ for $\tau_i$.
\\
According to Lemma 2 $\adv$ can compute $\hat{g}_i = g_1^{\frac{1}{x+\tau_i}}$.
\\
To complete the computation $\adv$ chooses  $\c_i,\s_i\sample\mathbb{Z}_p$ and computes
$$R = g_1^{\s_i }\cdot(X\cdot g_1^{\tau})^{-\c_i} $$
$$\hat{R} = \hat{g}^\s_i \cdot (g\cdot \hat{g}^{-\tau} )^\c_i.$$
Now $\adv$ returns $(\com_i = (\hat{g_i},R,\hat{R}),\c_i,\s_i)$ to $\fdv$ and this is indistinguishable from the normal transcript generation for $\fdv$ since if we define $r$ to be $r = \s-\c x$ then $R = g_1^x$ and 
$\hat{R} = \hat{g}^x$. Additionally since $\s$ and $\c$ are uniformly distributed in $\mathbb{Z}_p$ so is $r$.
\subsection*{Breaking the $q+1$-SDH:}
Eventually forger $\fdv$ returns a forgery $(\tau^*,\com^*,\c^*,\s^*)$ we assume that $\fdv$ wins the game and thus $\tau^* \notin \{\tau_1,...,\tau_q\}$ and 
$\IDAltV(\sk, \tau^*,\com^*,\c^*,\s^*) = 1$ which means if we parse $\com^*$ as $(\hat{g}^*,R^*,\hat{R}^*)$ 
$$\hat{g}^* = g_1^{\frac{1}{x+\tau^*}} = g^{\frac{\theta f(x)}{x+\tau^*}}$$
will hold.
\\
According to Lemma 2, $\adv$ can compute 
$ g^{\frac{1}{x+ \tau^*}}$
and ultimately return the pair $(\tau^*, g^{\frac{1}{x+ \tau^*}})$ as the solution to the $q+1-$SDH problem.

\end{proof}  