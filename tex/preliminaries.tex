\section{Preliminaries}\label{sec:preliminaries}

% \TODO{ALL THEOREM STATEMENT SHOULD INCLUDE RUNNING TIME BEING MORE OR LESS THE SAME.}

% In this section, we present important preliminaries.
% Additional definitions and security notions used throughout the paper including pseudorandom functions, non-interactive key exchange, and one-time signatures can be found in~\Cref{app:preliminaries}.




% We say a ring signature $\RSig = (\RSigGen, \RSigSgn, \RSigVer)$ randomness spaces $\R$ decomposing into $\R_1, \R_2$ such that $\R = \R_1 \times \R_2$ is
% \emph{randomness extractable} \jonas{change name maybe} if there exists an algorithm
% $
% \random_1 \gets Rec(\pk, \sigma)
% $
% which on input a public key and a signature outputs randomness $r_1 \in \R_1$ such that for all messages $\m$, $|\rho| = 2$ it holds
% $$
% \Pr_{(\random_1, \random_2) \sample \R_1 \times \R_2}[\RSigRec(\pk_s, \sigma) = \random_1 \mid (\sk_s, \pk_s), (\sk_r, \pk_r) \sample \RSigGen, \sigma \gets \RSigSgn(\sk_s, \{\pk_s, \pk_r \}, \m; (\random_1, \random_2))] \leq \negl.
% $$


\subsection{Signatures}

A digital signature scheme consists of three algorithms $(\SigGen, \SigSgn, \SigVer)$. A key generation algorithm $\SigGen$, a signing algorithm $\SigSgn$ and a verification algorithm $\SigVer$. $\SigGen$ is a randomised algorithm that produces a random key pair consisting of a public key $\pk$ and a secret key $\sk$. The probabilistic signing algorithm $\SigSgn$ requires a secret key and a message from the message space $\mathsf{M}$ and produces a signature $\sigma$. Finally, the verification algorithm $\SigVer$ takes a public key, a message and a signature as input and returns either 0/reject or 1/accept. A signature scheme is called correct, if every signature on a message generated with a secret key is accepted under the corresponding public key.

\begin{definition}[RMA security]\label{def:rma}
A signature scheme $\Pi = (\SigGen, \SigSgn, \SigVer)$ is said to be $(t,q,\epsilon)$-secure against existential forgery under the random message attack (EUF-RMA), if for all adversary $\adv$ running in time at most $t$ we have
$$\Pr[q\hyph\rma_{\Pi}(\adv)]=1] \leq \epsilon$$
where $q\hyph\rma_{\Pi}$ is depicted in \autoref{fig:rma}.


We say $\adv$, $(t,q,\epsilon)$-breaks the EUF-RMA security of the signature if
$$\Pr[q\hyph\rma_{\Pi}(\adv)=1] > \epsilon$$
\end{definition}



\begin{figure}[htb!]
    \centering
    
    \nicoresetlinenr
    \fbox{
        \begin{minipage}[t]{0.8\textwidth}
            \underline{$\textbf{Game } q\hyph\rma_{\Pi}(\adv)$}
            \begin{nicodemus}
            	\item $(\sk,\pk)\sample\SigGen$
		\item $\Q \leftarrow \emptyset$
                \item \pcfor $i \in [q]$
                \item \quad $ \m_i \sample \mathsf{M}$
                \item \quad $\sigma_i \sample \SigSgn(\sk,\m_i)$
                \item \quad $\Q \leftarrow \Q \cup (m_i,\sigma_i) $
                \item $(\m^*, \sigma^*)\gets\adv(\pk,\Q)$
                \item $\pcif \SigVer(\pk,\m^*,\sigma^*) = 1 \land m^* \notin \{m_1,...,m_q\}  \pcthen \pcreturn 1$
                \item $\pcelse \pcreturn 0$
            \end{nicodemus}
        \end{minipage}
    }
    \caption{}
    \label{fig:rma}
\end{figure}



\subsection{Hash Functions}
\aysan{Fix hash function dfinition.}
Let $\mathbb{G}= (\G_k)$ be a family of groups, indexed by the security parameter $k \in \mathbb{N}$ (we omit the subscript when the reference to the security parameter is clear, thus write $\G$ for $\G_k$).
\\
A \textit{group hash function} $\Hash$  over $\G$ with input length $l =l(k)$ consists of two efficient algorithms $\phfGen$ and $\phfEval$. The probabilistic algorithm 
$\kappa \sample \phfGen(1^k)$ generates a hash key $\kappa$ for the security parameter $k$. Algorithm 
$\phfEval$ is a deterministic algorithm, taking as input a hash function key $\kappa$ and $X \in \{0,1\}^l$, and returning $\phfEval(\kappa,X) \in \G$. In the context were $\kappa$ is clear we write 
$\phfEval(\kappa,X)$ as $\Hash(X)$.
\\

\begin{definition}[Correlation Interactibilty]
We say an adversary $\adv$, $(t, \epsilon)-$breaks the correlation intractability of a hash function  $\Hash = (\phfGen, \phfEval)$ with regards to function $g$ if $\adv$ runs in time $t$ and 
$$\Pr[x \sample \adv, \phfEval(\kappa, x) = g(x); \kappa \sample \phfGen] \geq \epsilon.$$
We call the hash function $(t, \epsilon)-$correlation intractable if such an adversary does not exist.
\end{definition}


\begin{definition}
A group hash function $H = (\phfGen, \phfEval)$ is a $(m,n,n\gamma,\delta)-$programmable, if there is an efficient trapdoor key generation algorithm $\phfTG$ and an efficient trapdoor evaluation algorithm $\phfTE$ with the following properties.
\begin{enumerate}
\item The probabilistic  trapdoor generation algorithm 
$(\kappa, \eta) \sample \phfTG(1^k,g_1,g_2)$ takes as input group elements 
$g,h \in \G$, and produces a hash function key $\kappa$ together with trapdoor information 
$\eta$.
\item For all generators
$g_1,g_2 \in \G$, the keys 
$\kappa \sample \phfGen(1^k)$ and 
$\kappa' \sample \phfGen(1^k,g_1,g_2)$ are statistically
$\gamma$-close.
\item 
On input $X \in \{0,1\}^l$ and trapdoor information $\eta$, the deterministic trapdoor evaluation algorithm
$(a_X,b_X) \leftarrow \phfTE(\eta,X)$ produces $a_X,b_X \in \mathbb{Z}$ so that for all 
$Xin \{0,1\}^l$,
$$\phfEval(\kappa, X) = g_1^{a_X}g_2^{b_X}.$$
\item For all $g_1,g_2 \in \G$, all $\kappa$ generated by 
$\kappa \sample \phfTG(1^k,g_1,g_2)$, and all 
$X_1,..,X_m \ in \{0,1\}^l$ and
$Z_1,...,Z_n \in \{0,1\}^l$ such that $X_i \neq Z_j$ for all $i,j$, we have
$$\Pr[a_{X_1}=...=a_{X_m}= 0 \land a_{Z_1},...,a_{Z_n} \neq 0]\geq \delta$$
where 
$(a_{X_i},b_{X_i}) = \phfTE(\eta, X_i)$ and $(a_{Z_i},b_{Z_i}) = \phfTG(\eta, Z_j)$, and the probability is taken over the trapdoor $\eta$ produced along with $\kappa$.
\end{enumerate}
\end{definition}











