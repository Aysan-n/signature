\section{Preliminaries}\label{sec:preliminaries}

% \TODO{ALL THEOREM STATEMENT SHOULD INCLUDE RUNNING TIME BEING MORE OR LESS THE SAME.}

% In this section, we present important preliminaries.
% Additional definitions and security notions used throughout the paper including pseudorandom functions, non-interactive key exchange, and one-time signatures can be found in~\Cref{app:preliminaries}.




% We say a ring signature $\RSig = (\RSigGen, \RSigSgn, \RSigVer)$ randomness spaces $\R$ decomposing into $\R_1, \R_2$ such that $\R = \R_1 \times \R_2$ is
% \emph{randomness extractable} \jonas{change name maybe} if there exists an algorithm
% $
% \random_1 \gets Rec(\pk, \sigma)
% $
% which on input a public key and a signature outputs randomness $r_1 \in \R_1$ such that for all messages $\m$, $|\rho| = 2$ it holds
% $$
% \Pr_{(\random_1, \random_2) \sample \R_1 \times \R_2}[\RSigRec(\pk_s, \sigma) = \random_1 \mid (\sk_s, \pk_s), (\sk_r, \pk_r) \sample \RSigGen, \sigma \gets \RSigSgn(\sk_s, \{\pk_s, \pk_r \}, \m; (\random_1, \random_2))] \leq \negl.
% $$


\begin{definition}[3-round Tag-based Identification Scheme]
A 3-round identification (ID) scheme is defined as $\ID \coloneqq(\IDGen, \IDP=(\IDP_1,\IDP_2) ,\ChSet, \IDV)$.
\begin{itemize}
\item The probabilistic generation algorithm $\IDGen$ takes the public parameter $1^k$ as input and returns a public key and secret key $(\pk,\sk)$. We assume that $\pk$ defines the challenge set $\ChSet$.

\item The prover algorithm $\IDP = (\IDP_1,\IDP_2)$ is split into two algorithms. $\IDP_1$ takes the secret key $\sk$ and a tag $\tau$ from the tag space $\mathsf{M}$ as the input and returns the commitment $\com$ and a state $\st$. $\IDP_2$ takes the secret key $\sk$, the state $\st$ and a challenge $\c$ as an input and returns a response $\s$. 

\item The deterministic verifier algorithm $\IDV$ takes the public key $\pk$, the tag $\tau$, the commitment $\com$, the challenge $\c$ and the response $\s$ as an input and outputs a decision, 1 (acceptance) or 0 (rejection).
\end{itemize}
For correctness we require that for all $k \in \mathbb{N}$,
$(\pk,\sk) \in \IDGen(1^k)$,
all
$(\com,\st) \in \IDP_1(\sk, \tau)$,
all $\c \in \ChSet$ and all 
$\s \in \IDP_2(\sk,\st,\c)$,
we have
$$ \IDV(\pk,\com,\c,\s) = 1.$$
\begin{figure}[H]
  \centering
   \nicoresetlinenr
\vspace{2em}
 \fbox{
 \pseudocode{%
 \textbf{ Prover} \< \< \textbf{ Verifier}  \\[0.1\baselineskip][\hline]
  \<\< \\[-0.5\baselineskip]
(\com,\st) \sample \IDP_1(\sk,\tau)
  \< \sendmessageright*{\tau,\com} \< \\[-2ex]
  \<\< \c \sample \ChSet \\[-4ex]
  \< \sendmessageleft*{\c} \< \\
  \s \sample\IDP_2(\sk,\st,\c) 
  \< \sendmessageright*{\s} \< \\[-2ex]
   \<\< d \coloneqq \IDV(\pk,\com,\c,\s)
  }
  
 }
  \caption{3-round Tag-based Identification Scheme}
 \end{figure}

\end{definition}


\begin{definition}[Alternative Verification]
We say the deterministic function $\IDAltV$ is an alternative verification for an identification scheme $\ID$, if $\IDAltV$ takes the secret key $\sk$, the tag $\tau$, the commitment $\com$, the challenge $\c$ and the response $\s$ as an input and outputs a decision, 1 (acceptance) or 0 (rejection).
\\
For correctness we require that for all $k \in \mathbb{N}$,
$(\pk,\sk) \in \IDGen(1^k)$,
all
$(\com,\st) \in \IDP_1(\sk, \tau)$,
all $\c \in \ChSet$ and all 
$\s \in \IDP_2(\sk,\st,\c)$,
we have
$$ \IDAltV(\sk,\tau,\com,\c,\s) = 1.$$

\end{definition}

\begin{definition}[Alternative Impersonation]
A 3-round tag based identification scheme is said to be $(t,q,\epsilon)-\imp^{\IDAltV}$ secure, if for all adversary $\adv$ running in time at most $t$ we have
$$\Pr[q\hyph\xsoundness^{\IDAltV}_{\ID}(\adv)]=1] \leq \epsilon.$$
\end{definition}



\begin{figure}[htb!]
    \centering
    \nicoresetlinenr
    \fbox{
        \begin{minipage}[t]{0.8\textwidth}
            \underline{$\textbf{Game } q\hyph\xsoundness^{\IDAltV}_{\ID}(\adv)$}
            \begin{nicodemus}
            	\item $(\sk,\pk)\sample\IDGen$
		\item $\Q \leftarrow \emptyset$
                \item \pcfor $i \in [q]$
                \item \quad $ \tau_i \sample \mathsf{M}$
                \item \quad $(\com_i,\st_i)\sample\IDP_1(\sk,\tau_i)$
                \item \quad $ \c_i \sample \ChSet $
                \item \quad $\s_i \sample\IDP_2(\sk,\st_i,\c_i)$
                \item \quad $\Q \leftarrow \Q \cup (\tau_i,\com_i,\c_i,\s_i) $
                \item $(\tau^*, \com^*,\c^*,\s^*)\gets\adv(\pk,\Q)$
                \item $\pcif \IDAltV(\sk, \tau^*, \com^*,\c^*,\s^*) = 1 \pcthen \pcreturn 1$
                \item $\pcelse \pcreturn 0$
            \end{nicodemus}
        \end{minipage}
    }
    \caption{}
    \label{fig:x-soundess}
\end{figure}




\begin{definition}[$\uniqueness$]
We say the identification scheme $\ID \coloneqq(\IDGen, \IDP ,\ChSet, \IDV)$ is
unique if for every $(\sk,\pk)\sample\IDGen$ and every $(\com,\st)\sample\IDP_1(\sk,\tau)$,
$$\big{|}\{\c \in \ChSet \mid \exists \s : \IDV(\pk,\com,\c,\s) = 1 \land \IDAltV(\sk, \com, \C,  \s) \neq 1 \}\big{|} = 1.$$
This means there exist a (not necessarily polynomial time) function we call the uniqueness function such as $f$ that
$$f(\pk,\com) = \c.$$
\end{definition}

\begin{definition}[Signature scheme] To construct a signature $\Sig \coloneqq (\SigGen, \SigSgn,
\SigVer )$from a 3-round tag-based identification scheme  $\ID \coloneqq(\IDGen, \IDP=(\IDP_1,\IDP_2) ,\ChSet, \IDV)$ we proceed as in Figure 4.

\begin{center}
\begin{figure}[htb!]
    \nicoresetlinenr
    \fbox{
       
        \begin{minipage}[t]{0.5\textwidth}
            \underline{$\SigGen(\pms)$:}
            \begin{nicodemus}
            \item $(\pk_0,\sk_0)\sample\IDGen$
            \item $(\pk_1,\sk_1)\sample\IDGen$
            \item $pk \coloneq (\pk_0,\pk_1)$
            \item $\sk \coloneq (\sk_0, \sk_1)$
            \item $\pcreturn (\sk,\pk)$
            \end{nicodemus}
            
              \underline{$\SigVer(\pk,\sigma,\m):$}
            \begin{nicodemus}
            	\item \pcif $\c_0+\c_1 \neq \Hash(\pk,\com_0,\com_1) $
		\item \quad \pcthen \pcreturn $0$
		\item \pcif $\IDV(\pk_0,\com_0,\c_0,\s_0) = 0$
		\item \quad \pcthen \pcreturn $0$
		\item \pcif $\IDV(\pk_1,\com_1,\c_1,\s_1) =0$
		\item \quad \pcthen \pcreturn $0$
		\item \pcelse \pcreturn $1$
            \end{nicodemus}
          
                       
        \end{minipage}
        \begin{minipage}[t]{0.5\textwidth}
            \underline{$\SigSgn(\sk,\m):$}
            \begin{nicodemus}
            	\item $(\sk_0,\sk_1) \leftarrow \sk$
            	\item	$(\com_0,\st_0)\sample\IDP_1(\sk_0,\m)$
		\item $(\com_1,\st_1)\sample\IDP_1(\sk_1,\m)$ 
		\item $ k = \Hash(\pk,\com_0,\com_1) $
  		\item $ e \sample \ChSet$
		\item $ \c_0 = d+e $
		\item $\c_1 = -e$  
		\item $\s_0 \sample\IDP_2(\sk_0,\st_0,\c_0)$
		\item $ \s_1 \sample\IDP_2(\sk_1,\st_1,\c_1)$
		\item $\sigma \coloneqq (\com_0,\c_0,\s_0, \com_1, \c_1,\s_1)$
		\item \pcreturn $\sigma$
            \end{nicodemus}

        \end{minipage}

    }
    \caption{Instantiation 1}
    \label{fig:prf-security}
\end{figure}
\end{center}
\end{definition}

\begin{definition}[RMA security]
We define the existential forgery against the random message attack (EUF-RMA) security experiment, played between a challenger and a forger $\mathcal{F}$.
\begin{enumerate}
\item The challenger runs $\SigGen$ to generate key pair $(\pk,\sk)$. 
The forger receives $\pk$ as input.
\item The challenger now chooses $q$ random messages and signs them and returns $(\m_i,\sigma_i)$ to the forger where $\sigma_i$ is $\m_i$ signed under $\sk$.
\item The forger outputs a message $\m^*$ and signature $\sigma^*$.
\end{enumerate}
$\mathcal{F}$ wins the game if $\SigVer(\pk,\sigma,\m) = 1$, that is, $\sigma^*$ is a valid signature for $\m^*$, and $\m^* \neq \m_i$ for all $i$. We say $\mathcal{F}$, $(t,q,\epsilon)$-breaks the EUF-RMA security of the signature, if $\mathcal{F}$ runs in time $t$, receives at most $q$ signed messages, and has the success probability of $\epsilon$. 
\end{definition}


\begin{definition}[Correlation Interactibilty]
We say an adversary $\adv$, $(t, \epsilon)-$breaks the correlation intractability of a hash function  $\Hash: \{0,1\}^{n(\lambda)} \rightarrow \{0,1\}^{m(\lambda)}$ with regards to function $g$ if $\adv$ runs in time $t$ and 
$$\Pr[x \sample \adv, \Hash(x) = g(x)] \geq \epsilon(\lambda).$$
We call the hash function $(t, \epsilon)-$correlation intractable if such an adversary does not exist.
\end{definition}




\subsection{proof}
\begin{theorem}
Let  $\ID$ be a unique identification scheme and $\Hash$ be a $(t'',\epsilon'')$ correlation intractable hash function. Suppose there exists a $(t,q,\epsilon)$-forger $\mathcal{F}$ breaking the security of $\Sig_{\ID,\Hash}$ against the existential forgery under the random message attack. Then there exists an adversary that $(t',q,\epsilon')$-breaks the $\imp^{\IDAltV}$ security of $\ID$ with $t' \approx t$ and
$$ \epsilon' \geq \frac{3}{4}\epsilon .$$
\end{theorem}
This results follows from Lemma 1 and 2 and a hybrid argument.

\begin{definition}[Partially valid signature]
signature  $\sigma = (\com_{0},\c_{0},\s_{0}, \com_{1}, \c_{1},\s_{1})$ is
partially valid if $\IDAltV(\pk_0,\com_0,\c_0,\s_0) =1$ or $\IDAltV(\pk_1,\com_1,\c_1,\s_1) = 1$
not partially valid if 
 $\IDAltV(\pk_0,\com_0,\c_0,\s_0) =0$ and $\IDAltV(\pk_1,\com_1,\c_1,\s_1) =0$.

\end{definition}

Let $(\m_i,\sigma_i)$ denote the $i$-th random message and it's signature. Let $(\m^*,\sigma^*)$ be the forgery output by $\mathcal{F}$.
\\
We distinguish between Type I forger returning $(\m^*, \sigma^*)$ where $(\m^*, \sigma^*)$ is patially valid  and Type II forger returning  $(\m^*, \sigma^*)$ where $(\m^*, \sigma^*)$ is not partially valid.
\subsubsection{Type I forger} 
\begin{lemma}
Let $\mathcal{F}$ be a type I forger that $(t,q,\epsilon)$-breaks the RMA security of the signature. Then there exists adversary $\mathcal{A}$ that $(t',q,\epsilon')$-breaks the $\imp^{\IDAltV}$ security of the ID scheme with 
$t \approx t'$ and 
$$\epsilon' \geq \frac{1}{2} \epsilon.$$
\end{lemma}

\subsection*{Game 0.}
We define Game 0 as the existential unforgeability experiment with forger $\mathcal{F}$. By definition, we have
$$\Pr[X_0] = \epsilon.$$

\subsection*{Game 1.}
We modify this game such that the game chooses a random bit $b$ at the beginning. When $\mathcal{F}$ outputs a forgery  $(\m^*, \sigma^*)$ the game parses the signature as 
$$(\com^*_{0},\c^*_{0},\s^*_{0}, \com^*_{1}, \c^*_{1},\s^*_{1})$$
 and aborts if 
$\IDAltV(\pk_b,\com^*_b,\c^*_b,\s^*_b) =1$. We denote this even with $\abort$.
\\
Since the forger is of type I and outputs a partially valid signature, we have
$\Pr[\abort] \leq \frac{1}{2}$, which implies
$$\Pr[X_1] = \Pr[X_0 \land \neg \abort] \geq \frac{1}{2} \Pr[X_0].$$

\subsection*{Game 2.}
In this Game we change the way signatures are calculate.
The game first signs every signature as before then changes every signature $\sigma_i = (\com_{0,i},\c_{0,i},\s_{0,i}, \com_{1,i}, \c_{1,i},\s_{1,i})$ for message $m_i$ as follows.
$$(\com_{1-b,i},\st_{1-b,i})\sample\IDP_1(\sk_{1-b},\m_i)$$
$$ k_i = \Hash(\pk,\com_{0,i},\com_{1,i}) $$
$$\c_{1-b,i} = k_i - \c_{b,i} $$
$$\s_{1-b,i} \sample\IDP_2(\sk_{1-b},\st_{1-b,i},\c_{1-b,i})$$
Finally the game returns the newly calculated signature $\sigma_i$ to $\mathcal{F}$.
\\
Game 2 is perfectly indistinguishable from game 1 from the adversary's point of view. Thus we have
$$\Pr[X_2] = \Pr[X_1].$$

\subsection*{The Alternative Impersonation Adversary.}
Now adversary $\adv$ simulates game 2. The $\adv$ receives $pk_{b}$ and $(\tau_i, \com_{b,i}, S_{i,b})$ from the alternative impersonation game and proceeds to calculate the public key and signatures on message $m_i := \tau_i$ as in game 2.
\\
It remains to show how $\adv$ can break the alternative impersonation from the forged signature
$\sigma^* = (\com^*_{0},\c^*_{0},\s^*_{0}, \com^*_{1}, \c^*_{1},\s^*_{1})$ on message $m^*$ output by $\mathcal{F}$. We know that $\IDAltV(\sk_b,m^*,\com^*_b,\c^*_b,\s^*_b) =1$ (by game 1).  So $\adv$ can win the alternative impersonation game by outputing $(\m^*,\com^*_b,\c^*_b,\s^*_b)$.

\subsubsection{Type II forger} 
\begin{lemma}
Let $\mathcal{F}$ be a type II forger that $(t,q,\epsilon)$-breaks the RMA security of the signature. Then there exists adversary $\mathcal{A}$ that $(t',\epsilon')$-breaks the correlation intractability of the hash function $\Hash$ with 
$t \approx t'$ and 
$$\epsilon' \geq \epsilon.$$
\end{lemma}

The correlation intractability adversary $\adv$ simulates the unforgeability experiment by
running $\IDGen$ twice and obtaining two pairs of keys we name $(\sk_0,\pk_0)$ and $(\sk_1, \pk_1)$. The adversary now return 
$\pk := (\pk_0, \pk_1)$ to $\mathcal{F}$ as the public key and also
chooses random messages $m_1,...,m_q$ and signs them with the secret key 
$\sk := (\sk_0, \sk_1)$ to obtain the signatures $\sigma_1,...,\sigma_q$ and returns the $(m_i,\sigma_i)$ pairs to $\mathcal{F}$. This game is indistinguishable from the unforgeability game in the view of $\mathcal{F}$.

\subsection*{Breaking the hash intractability.}
Eventually, $\mathcal{F}$ returns a message and signature pair 
 $(\m^*, \sigma^*)$, from which $\adv$ extracts the solution that breaks the hash intractability as follows. First $\adv$ parses $\sigma^*$
as
$(\com^*_{0},\c^*_{0},\s^*_{0}, \com^*_{1}, \c^*_{1},\s^*_{1})$.
We assume the signature is valid and because forger type II outputs signatures that are not partially valid, due to the uniqueness of the identification scheme
 we can write
$$\c^*_0 = f(\pk_0,\com^*_0)$$
$$\c^*_1 = f(\pk_1,\com^*_1).$$
If the forged signature is valid then 
$$\Hash(\pk,\com^*_0,\com^*_1)=\c^*_0+\c^*_1 =  f(\pk_0,\com^*_0)+ f(\pk_1,\com^*_1) = g(\pk,\com^*_0,\com^*_1).$$
So $\adv$ can $(t,\epsilon')$ break the $g$-correlation intractability of $\Hash$ where $g$ is defined as 
$$g(\pk=(\pk_0,\pk_1), \com_0, \com_1) =  f(\pk_0,\com_0)+ f(\pk_1,\com_1).$$
Adversary $\adv$ succeeds at giving a solution that breaks the correlation intractability of $\Hash$ whenever $\mathcal{F}$ succeeds at forging a valid signature so
$$\epsilon'\geq\epsilon.$$



\subsection{Instantiation from the q-SDH}
In the following let $\pms \coloneqq (p,q,\G) $ be a set of system parameters, where 
$\G = <g>$ is a cyclic group of prime order $p$.

\begin{definition}[q-SDH]
\TODO{}
\end{definition}

\begin{figure}[htb!]
    \centering
    \nicoresetlinenr
    \fbox{
        \begin{minipage}[t]{0.5\textwidth}
            \underline{$\IDGen(\pms)$:}
            \begin{nicodemus}
                \item $\sk \coloneqq x \sample \mathbb{Z}_p$
                \item $\pk \coloneqq y = g^x$
                \item $\ChSet \coloneqq \mathbb{Z}_p$
                \item $\pcreturn (\sk,\pk)$
            \end{nicodemus}
            \underline{$\IDV(\pk,\com,\c,\s)$}:
            \begin{nicodemus}
                \item \pcparse $R \coloneqq (h,u,\hat{u})$
                \item \pcif $u = g^\s \cdot(y\cdot g^{\m})^{-c} \land \hat{u} = h^\s \cdot g^{-c} $ 
                \item \quad \pcthen \pcreturn 1
                \item \pcelse \pcreturn 0
            \end{nicodemus}
            
            \underline{$\IDAltV(\sk,\m,\com, \?{})$}:
            \begin{nicodemus}
                \item \pcparse $R \coloneqq (h,u,\hat{u}), \sk = x$
                \item \pcif $h = g^{\frac{1}{x+m}}$ 
                \item \quad \pcthen \pcreturn 1
                \item \pcelse \pcreturn 0
            \end{nicodemus}
            
        \end{minipage}
        \begin{minipage}[t]{0.5\textwidth}
            \underline{$\IDP_1(\sk,m):$}
            \begin{nicodemus}
                \item $\st \coloneqq r \sample \mathbb{Z}_p$
                \item $h \coloneqq g^{\frac{1}{x+m}}$
                \item $u \coloneqq g^r$
                \item $\hat{u} \coloneqq h^r$
                \item $R = (h,u,\hat{u})$
                \item $\pcreturn (\com,\st)$
            \end{nicodemus}
            
              \underline{$\IDP_2(\sk,\st,\c):$}
            \begin{nicodemus}
            	\item $\pcparse \st = r$
                \item $\pcreturn \s = c\cdot(x+m) + r \mod p$
            \end{nicodemus}
        \end{minipage}
    }
    \caption{Instantiation 1}
    \label{fig:prf-security}
\end{figure}

We describe the identification scheme  $\ID \coloneqq(\IDGen, \IDP=(\IDP_1,\IDP_2) ,\ChSet, \IDV)$ as the following
\begin{itemize}[label={}]
  \item $\IDGen(1^k):$ Let $g$ be a random generator of $\G$. The private key is $x \sample \mathbb{Z}_q$ and the public key is $y = g^x$.
  \item $\IDP=(\IDP_1,\IDP_2)$
  \item $\IDV$
\end{itemize}



