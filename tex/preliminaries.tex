\section{Preliminaries}\label{sec:preliminaries}

% \TODO{ALL THEOREM STATEMENT SHOULD INCLUDE RUNNING TIME BEING MORE OR LESS THE SAME.}

% In this section, we present important preliminaries.
% Additional definitions and security notions used throughout the paper including pseudorandom functions, non-interactive key exchange, and one-time signatures can be found in~\Cref{app:preliminaries}.




% We say a ring signature $\RSig = (\RSigGen, \RSigSgn, \RSigVer)$ randomness spaces $\R$ decomposing into $\R_1, \R_2$ such that $\R = \R_1 \times \R_2$ is
% \emph{randomness extractable} \jonas{change name maybe} if there exists an algorithm
% $
% \random_1 \gets Rec(\pk, \sigma)
% $
% which on input a public key and a signature outputs randomness $r_1 \in \R_1$ such that for all messages $\m$, $|\rho| = 2$ it holds
% $$
% \Pr_{(\random_1, \random_2) \sample \R_1 \times \R_2}[\RSigRec(\pk_s, \sigma) = \random_1 \mid (\sk_s, \pk_s), (\sk_r, \pk_r) \sample \RSigGen, \sigma \gets \RSigSgn(\sk_s, \{\pk_s, \pk_r \}, \m; (\random_1, \random_2))] \leq \negl.
% $$


\begin{definition}[3-round Tag-based Identification Scheme]
A 3-round identification (ID) scheme is defined as $\ID \coloneqq(\IDGen, \IDP=(\IDP_1,\IDP_2) ,\ChSet, \IDV)$.
\begin{itemize}
\item The probabilistic generation algorithm $\IDGen$ takes the public parameter $1^k$ as input and returns a public key and secret key $(\pk,\sk)$. We assume that $\pk$ defines the challenge set $\ChSet$.

\item The prover algorithm $\IDP = (\IDP_1,\IDP_2)$ is split into two algorithms. $\IDP_1$ takes the secret key $\sk$ and a tag $\tau$ from the tag space $\mathsf{M}$ as the input and returns the commitment $\com$ and a state $\st$. $\IDP_2$ takes the secret key $\sk$, the state $\st$ and a challenge $\c$ as an input and returns a response $\s$. 

\item The deterministic verifier algorithm $\IDV$ takes the public key $\pk$, the tag $\tau$, the commitment $\com$, the challenge $\c$ and the response $\s$ as an input and outputs a decision, 1 (acceptance) or 0 (rejection).
\end{itemize}
For correctness we require that for all $k \in \mathbb{N}$,
$(\pk,\sk) \in \IDGen(1^k)$,
all
$(\com,\st) \in \IDP_1(\sk, \tau)$,
all $\c \in \ChSet$ and all 
$\s \in \IDP_2(\sk,\st,\c)$,
we have
$$ \IDV(\pk,\com,\c,\s) = 1.$$
\begin{figure}[H]
  \centering
   \nicoresetlinenr
\vspace{2em}
 \fbox{
 \pseudocode{%
 \textbf{ Prover} \< \< \textbf{ Verifier}  \\[0.1\baselineskip][\hline]
  \<\< \\[-0.5\baselineskip]
(\com,\st) \sample \IDP_1(\sk,\tau)
  \< \sendmessageright*{\tau,\com} \< \\[-2ex]
  \<\< \c \sample \ChSet \\[-4ex]
  \< \sendmessageleft*{\c} \< \\
  \s \sample\IDP_2(\sk,\st,\c) 
  \< \sendmessageright*{\s} \< \\[-2ex]
   \<\< d \coloneqq \IDV(\pk,\com,\c,\s)
  }
  
 }
  \caption{3-round Tag-based Identification Scheme}
 \end{figure}

\end{definition}


\begin{definition}[Alternative Verification]
We say the deterministic function $\IDAltV$ is an alternative verification for an identification scheme $\ID$, if $\IDAltV$ takes the secret key $\sk$, the tag $\tau$, the commitment $\com$, the challenge $\c$ and the response $\s$ as an input and outputs a decision, 1 (acceptance) or 0 (rejection).
\\
For correctness we require that for all $k \in \mathbb{N}$,
$(\pk,\sk) \in \IDGen(1^k)$,
all
$(\com,\st) \in \IDP_1(\sk, \tau)$,
all $\c \in \ChSet$ and all 
$\s \in \IDP_2(\sk,\st,\c)$,
we have
$$ \IDAltV(\sk,\tau,\com,\c,\s) = 1.$$

\end{definition}

\begin{definition}[Alternative Impersonation]
A 3-round tag based identification scheme is said to be $(t,q,\epsilon)-\imp^{\IDAltV}$ secure, if for all adversary $\adv$ running in time at most $t$ we have
$$\Pr[q\hyph\xsoundness^{\IDAltV}_{\ID}(\adv)]=1] \leq \epsilon.$$
\end{definition}



\begin{figure}[htb!]
    \centering
    \nicoresetlinenr
    \fbox{
        \begin{minipage}[t]{0.8\textwidth}
            \underline{$\textbf{Game } q\hyph\xsoundness^{\IDAltV}_{\ID}(\adv)$}
            \begin{nicodemus}
            	\item $(\sk,\pk)\sample\IDGen$
		\item $\Q \leftarrow \emptyset$
                \item \pcfor $i \in [q]$
                \item \quad $ \tau_i \sample \mathsf{M}$
                \item \quad $(\com_i,\st_i)\sample\IDP_1(\sk,\tau_i)$
                \item \quad $ \c_i \sample \ChSet $
                \item \quad $\s_i \sample\IDP_2(\sk,\st_i,\c_i)$
                \item \quad $\Q \leftarrow \Q \cup (\tau_i,\com_i,\c_i,\s_i) $
                \item $(\tau^*, \com^*,\c^*,\s^*)\gets\adv(\pk,\Q)$
                \item $\pcif \IDAltV(\sk, \tau^*, \com^*,\c^*,\s^*) = 1 \pcthen \pcreturn 1$
                \item $\pcelse \pcreturn 0$
            \end{nicodemus}
        \end{minipage}
    }
    \caption{}
    \label{fig:x-soundess}
\end{figure}




\begin{definition}[$\uniqueness$]
We say the identification scheme $\ID \coloneqq(\IDGen, \IDP ,\ChSet, \IDV)$ is
unique if for every $(\sk,\pk)\sample\IDGen$ and every $(\com,\st)\sample\IDP_1(\sk,\tau)$,
$$\big{|}\{\c \in \ChSet \mid \exists \s : \IDV(\pk,\com,\c,\s) = 1 \land \IDAltV(\sk, \com, \C,  \s) \neq 1 \}\big{|} = 1.$$
This means there exist a (not necessarily polynomial time) function we call the uniqueness function such as $f$ that
$$f(\pk,\com) = \c.$$
\end{definition}


\begin{definition}[RMA security]
We define the existential forgery against the random message attack (EUF-RMA) security experiment, played between a challenger and a forger $\mathcal{F}$.
\begin{enumerate}
\item The challenger runs $\SigGen$ to generate key pair $(\pk,\sk)$. 
The forger receives $\pk$ as input.
\item The challenger now chooses $q$ random messages and signs them and returns $(\m_i,\sigma_i)$ to the forger where $\sigma_i$ is $\m_i$ signed under $\sk$.
\item The forger outputs a message $\m^*$ and signature $\sigma^*$.
\end{enumerate}
$\mathcal{F}$ wins the game if $\SigVer(\pk,\sigma,\m) = 1$, that is, $\sigma^*$ is a valid signature for $\m^*$, and $\m^* \neq \m_i$ for all $i$. We say $\mathcal{F}$, $(t,q,\epsilon)$-breaks the EUF-RMA security of the signature, if $\mathcal{F}$ runs in time $t$, receives at most $q$ signed messages, and has the success probability of $\epsilon$. 
\end{definition}


\begin{definition}[Correlation Interactibilty]
We say an adversary $\adv$, $(t, \epsilon)-$breaks the correlation intractability of a hash function  $\Hash: \{0,1\}^{n(\lambda)} \rightarrow \{0,1\}^{m(\lambda)}$ with regards to function $g$ if $\adv$ runs in time $t$ and 
$$\Pr[x \sample \adv, \Hash(x) = g(x)] \geq \epsilon(\lambda).$$
We call the hash function $(t, \epsilon)-$correlation intractable if such an adversary does not exist.
\end{definition}







