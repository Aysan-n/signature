\section{Constructions}

\begin{definition}[Signature scheme] To construct a signature $\Sig \coloneqq (\SigGen, \SigSgn,
\SigVer )$from a 3-round tag-based identification scheme  $\ID \coloneqq(\IDGen, \IDP=(\IDP_1,\IDP_2) ,\ChSet, \IDV)$ we proceed as in Figure 4.

\begin{center}
\begin{figure}[htb!]
    \nicoresetlinenr
    \fbox{
       
        \begin{minipage}[t]{0.5\textwidth}
            \underline{$\SigGen(\pms)$:}
            \begin{nicodemus}
            \item $(\pk_0,\sk_0)\sample\IDGen$
            \item $(\pk_1,\sk_1)\sample\IDGen$
            \item $pk \coloneq (\pk_0,\pk_1)$
            \item $\sk \coloneq (\sk_0, \sk_1)$
            \item $\pcreturn (\sk,\pk)$
            \end{nicodemus}
            
              \underline{$\SigVer(\pk,\sigma,\m):$}
            \begin{nicodemus}
            	\item \pcif $\c_0+\c_1 \neq \Hash(\pk,\com_0,\com_1) $
		\item \quad \pcthen \pcreturn $0$
		\item \pcif $\IDV(\pk_0,\com_0,\c_0,\s_0) = 0$
		\item \quad \pcthen \pcreturn $0$
		\item \pcif $\IDV(\pk_1,\com_1,\c_1,\s_1) =0$
		\item \quad \pcthen \pcreturn $0$
		\item \pcelse \pcreturn $1$
            \end{nicodemus}
          
                       
        \end{minipage}
        \begin{minipage}[t]{0.5\textwidth}
            \underline{$\SigSgn(\sk,\m):$}
            \begin{nicodemus}
            	\item $\pcparse \sk = (\sk_0,\sk_1) $
            	\item	$(\com_0,\st_0)\sample\IDP_1(\sk_0,\m)$
		\item $(\com_1,\st_1)\sample\IDP_1(\sk_1,\m)$ 
		\item $ k = \Hash(\pk,\com_0,\com_1) $
  		\item $ e \sample \ChSet$
		\item $ \c_0 = d+e $
		\item $\c_1 = -e$  
		\item $\s_0 \sample\IDP_2(\sk_0,\st_0,\c_0)$
		\item $ \s_1 \sample\IDP_2(\sk_1,\st_1,\c_1)$
		\item $\sigma \coloneqq (\com_0,\c_0,\s_0, \com_1, \c_1,\s_1)$
		\item \pcreturn $\sigma$
            \end{nicodemus}

        \end{minipage}

    }
    \caption{Instantiation 1}
    \label{fig:prf-security}
\end{figure}
\end{center}
\end{definition}


\subsection{Security}

\begin{theorem}
Let  $\ID$ be a unique identification scheme and $\Hash$ be a $(t'',\epsilon'')$ correlation intractable hash function. Suppose there exists a $(t,q,\epsilon)$-forger $\mathcal{F}$ breaking the security of $\Sig_{\ID,\Hash}$ against the existential forgery under the random message attack. Then there exists an adversary that $(t',q,\epsilon')$-breaks the $\imp^{\IDAltV}$ security of $\ID$ with $t' \approx t$ and
$$ \epsilon' \geq \frac{3}{4}\epsilon .$$
\end{theorem}
\begin{proof}
This results follows from Lemma 1 and 2 and a hybrid argument.

\begin{definition}[Partially valid signature]
signature  $\sigma = (\com_{0},\c_{0},\s_{0}, \com_{1}, \c_{1},\s_{1})$ is
partially valid if $\IDAltV(\pk_0,\com_0,\c_0,\s_0) =1$ or $\IDAltV(\pk_1,\com_1,\c_1,\s_1) = 1$
not partially valid if 
 $\IDAltV(\pk_0,\com_0,\c_0,\s_0) =0$ and $\IDAltV(\pk_1,\com_1,\c_1,\s_1) =0$.

\end{definition}

Let $(\m_i,\sigma_i)$ denote the $i$-th random message and it's signature. Let $(\m^*,\sigma^*)$ be the forgery output by $\mathcal{F}$.
\\
We distinguish between Type I forger returning $(\m^*, \sigma^*)$ where $(\m^*, \sigma^*)$ is patially valid  and Type II forger returning  $(\m^*, \sigma^*)$ where $(\m^*, \sigma^*)$ is not partially valid.
\subsubsection{Type I forger} 
\begin{lemma}
Let $\mathcal{F}$ be a type I forger that $(t,q,\epsilon)$-breaks the RMA security of the signature. Then there exists adversary $\mathcal{A}$ that $(t',q,\epsilon')$-breaks the $\imp^{\IDAltV}$ security of the ID scheme with 
$t \approx t'$ and 
$$\epsilon' \geq \frac{1}{2} \epsilon.$$
\end{lemma}

\subsection*{Game 0.}
We define Game 0 as the existential unforgeability experiment with forger $\mathcal{F}$. By definition, we have
$$\Pr[X_0] = \epsilon.$$

\subsection*{Game 1.}
We modify this game such that the game chooses a random bit $b$ at the beginning. When $\mathcal{F}$ outputs a forgery  $(\m^*, \sigma^*)$ the game parses the signature as 
$$(\com^*_{0},\c^*_{0},\s^*_{0}, \com^*_{1}, \c^*_{1},\s^*_{1})$$
 and aborts if 
$\IDAltV(\pk_b,\com^*_b,\c^*_b,\s^*_b) =1$. We denote this even with $\abort$.
\\
Since the forger is of type I and outputs a partially valid signature, we have
$\Pr[\abort] \leq \frac{1}{2}$, which implies
$$\Pr[X_1] = \Pr[X_0 \land \neg \abort] \geq \frac{1}{2} \Pr[X_0].$$

\subsection*{Game 2.}
In this Game we change the way signatures are calculate.
The game first signs every signature as before then changes every signature $\sigma_i = (\com_{0,i},\c_{0,i},\s_{0,i}, \com_{1,i}, \c_{1,i},\s_{1,i})$ for message $m_i$ as follows.
$$(\com_{1-b,i},\st_{1-b,i})\sample\IDP_1(\sk_{1-b},\m_i)$$
$$ k_i = \Hash(\pk,\com_{0,i},\com_{1,i}) $$
$$\c_{1-b,i} = k_i - \c_{b,i} $$
$$\s_{1-b,i} \sample\IDP_2(\sk_{1-b},\st_{1-b,i},\c_{1-b,i})$$
Finally the game returns the newly calculated signature $\sigma_i$ to $\mathcal{F}$.
\\
Game 2 is perfectly indistinguishable from game 1 from the adversary's point of view. Thus we have
$$\Pr[X_2] = \Pr[X_1].$$

\subsection*{The Alternative Impersonation Adversary.}
Now adversary $\adv$ simulates game 2. The $\adv$ receives $pk_{b}$ and $(\tau_i, \com_{b,i}, S_{i,b})$ from the alternative impersonation game and proceeds to calculate the public key and signatures on message $m_i := \tau_i$ as in game 2.
\\
It remains to show how $\adv$ can break the alternative impersonation from the forged signature
$\sigma^* = (\com^*_{0},\c^*_{0},\s^*_{0}, \com^*_{1}, \c^*_{1},\s^*_{1})$ on message $m^*$ output by $\mathcal{F}$. We know that $\IDAltV(\sk_b,m^*,\com^*_b,\c^*_b,\s^*_b) =1$ (by game 1).  So $\adv$ can win the alternative impersonation game by outputing $(\m^*,\com^*_b,\c^*_b,\s^*_b)$.

\subsubsection{Type II forger} 
\begin{lemma}
Let $\mathcal{F}$ be a type II forger that $(t,q,\epsilon)$-breaks the RMA security of the signature. Then there exists adversary $\mathcal{A}$ that $(t',\epsilon')$-breaks the correlation intractability of the hash function $\Hash$ with 
$t \approx t'$ and 
$$\epsilon' \geq \epsilon.$$
\end{lemma}

The correlation intractability adversary $\adv$ simulates the unforgeability experiment by
running $\IDGen$ twice and obtaining two pairs of keys we name $(\sk_0,\pk_0)$ and $(\sk_1, \pk_1)$. The adversary now return 
$\pk := (\pk_0, \pk_1)$ to $\mathcal{F}$ as the public key and also
chooses random messages $m_1,...,m_q$ and signs them with the secret key 
$\sk := (\sk_0, \sk_1)$ to obtain the signatures $\sigma_1,...,\sigma_q$ and returns the $(m_i,\sigma_i)$ pairs to $\mathcal{F}$. This game is indistinguishable from the unforgeability game in the view of $\mathcal{F}$.

\subsection*{Breaking the hash intractability.}
Eventually, $\mathcal{F}$ returns a message and signature pair 
 $(\m^*, \sigma^*)$, from which $\adv$ extracts the solution that breaks the hash intractability as follows. First $\adv$ parses $\sigma^*$
as
$(\com^*_{0},\c^*_{0},\s^*_{0}, \com^*_{1}, \c^*_{1},\s^*_{1})$.
We assume the signature is valid and because forger type II outputs signatures that are not partially valid, due to the uniqueness of the identification scheme
 we can write
$$\c^*_0 = f(\pk_0,\com^*_0)$$
$$\c^*_1 = f(\pk_1,\com^*_1).$$
If the forged signature is valid then 
$$\Hash(\pk,\com^*_0,\com^*_1)=\c^*_0+\c^*_1 =  f(\pk_0,\com^*_0)+ f(\pk_1,\com^*_1) = g(\pk,\com^*_0,\com^*_1).$$
So $\adv$ can $(t,\epsilon')$ break the $g$-correlation intractability of $\Hash$ where $g$ is defined as 
$$g(\pk=(\pk_0,\pk_1), \com_0, \com_1) =  f(\pk_0,\com_0)+ f(\pk_1,\com_1).$$
Adversary $\adv$ succeeds at giving a solution that breaks the correlation intractability of $\Hash$ whenever $\mathcal{F}$ succeeds at forging a valid signature so
$$\epsilon'\geq\epsilon.$$

\end{proof}



