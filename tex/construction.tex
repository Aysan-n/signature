\section{Constructions}

\begin{definition}[Signature scheme] To construct a signature $\Sig \coloneqq (\SigGen, \SigSgn,
\SigVer )$ from a 3-round tag-based identification scheme  $\ID \coloneqq(\IDGen, \IDP, \IDV)$ we proceed as in Figure 4.

\begin{center}
\begin{figure}[htb!]
    \nicoresetlinenr
    \fbox{
       
        \begin{minipage}[t]{0.51\textwidth}
            \underline{$\SigGen(\pms)$:}
            \begin{nicodemus}
            \item $(\pk_0,\sk_0)\sample\IDGen$
            \item $(\pk_1,\sk_1)\sample\IDGen$
            \item $pk \coloneqq (\pk_0,\pk_1)$
            \item $\sk \coloneqq (\sk_0, \sk_1)$
            \item $\pcreturn (\sk,\pk)$
            \end{nicodemus}
            
              \underline{$\SigVer(\pk,\sigma,\m):$}
            \begin{nicodemus}
            	\item \pcparse $\sigma = (\com_0,\c_0,\s_0, \com_1, \c_1,\s_1)$
            	\item \pcif $\c_0\oplus \c_1 \neq \Hash(\pk,\com_0,\com_1) $
		\item \quad \pcthen \pcreturn $0$
		\item \pcif $\IDV(\pk_0,\com_0,\c_0,\s_0) = 0$
		\item \quad \pcthen \pcreturn $0$
		\item \pcif $\IDV(\pk_1,\com_1,\c_1,\s_1) =0$
		\item \quad \pcthen \pcreturn $0$
		\item \pcelse \pcreturn $1$
            \end{nicodemus}
          
                       
        \end{minipage}
        \begin{minipage}[t]{0.5\textwidth}
            \underline{$\SigSgn(\sk,\m):$}
            \begin{nicodemus}
            	\item $\pcparse \sk = (\sk_0,\sk_1) $
            	\item	$(\com_0,\st_0)\sample\IDP_1(\sk_0,\m)$
		\item $(\com_1,\st_1)\sample\IDP_1(\sk_1,\m)$ 
		\item $ k = \Hash(\pk,\com_0,\com_1) $
  		\item $ e \sample \ChSet$
		\item $ \c_0 = d\oplus e $
		\item $\c_1 = e$  
		\item $\s_0 \sample\IDP_2(\sk_0,\st_0,\c_0)$
		\item $ \s_1 \sample\IDP_2(\sk_1,\st_1,\c_1)$
		\item $\sigma \coloneqq (\com_0,\c_0,\s_0, \com_1, \c_1,\s_1)$
		\item \pcreturn $\sigma$
            \end{nicodemus}

        \end{minipage}

    }
    \caption{Instantiation 1}
    \label{fig:prf-security}
\end{figure}
\end{center}
\end{definition}

\begin{definition}[Partially valid signature]
signature  $\sigma = (\com_{0},\c_{0},\s_{0}, \com_{1}, \c_{1},\s_{1})$ is
partially valid if $\IDAltV(\pk_0,\com_0,\c_0,\s_0) =1$ or $\IDAltV(\pk_1,\com_1,\c_1,\s_1) = 1$
not partially valid if 
 $\IDAltV(\pk_0,\com_0,\c_0,\s_0) =0$ and $\IDAltV(\pk_1,\com_1,\c_1,\s_1) =0$.

\end{definition}
\subsection{Security}
\begin{figure}[htb!]
    \centering
    \nicoresetlinenr
    \resizebox{\textwidth}{!}{\fbox{
        \begin{minipage}[t]{0.8\textwidth}
            \underline{$\G_0-\G_3$}
            \begin{nicodemus}
            \item $b \sample \{0,1\}$
            \hfill\comment{$\G_2-\G_3 $}
            \item $\BAD_2 \leftarrow \true$
            \hfill\comment{$\G_2-\G_3 $}

            \item $(\pk_0,\sk_0)\sample\IDGen$
            \item $(\pk_1,\sk_1)\sample\IDGen$
            \item $pk \coloneqq (\pk_0,\pk_1)$
            \item $\sk \coloneqq (\sk_0, \sk_1)$
            \item \pcfor $i \in [q]$
                \item \quad $ \m_i \sample \mathsf{M}$
            	\item	\quad $(\com_{0,i},\st_{0,i})\sample\IDP_1(\sk_0,\m_i)$
		\item \quad $(\com_{1,i},\st_{1,i})\sample\IDP_1(\sk_1,\m_i)$ 
		\item \quad $ k_i = \Hash(\pk,\com_{0,i},\com_{1,i}) $
  		\item \quad $ e_i \sample \ChSet$
		\item \quad $ \c_{0,i} = k_i \oplus e_i $
		\item \quad $\c_{1,i} = e_i$  
		\item \quad $\s_{0,i} \sample\IDP_2(\sk_0,\st_{0,i},\c_{0,i})$
		\item \quad $ \s_{1,i} \sample\IDP_2(\sk_1,\st_{1,i},\c_{1,i})$
		\item \quad $\sigma_i \coloneqq (\com_{0,i},\c_{0,i},\s_{0,i}, \com_{1,i}, \c_{1,i},\s_{1,i})$
                \item \quad $\Q \leftarrow \Q \cup (m_i,\sigma_i) $
                \item $(\m*, \sigma^*)\gets\adv(\pk,\Q)$
                \item \pcparse $\sigma^* = (\com_0^*,\c_0^*,\s_0^*, \com_1^*, \c_1^*,\s_1^*)$
                \item \pcif $\IDAltV(\pk_0,\com^*_0,\c^*_0,\s^*_0) = 0 \land \IDAltV(\pk_0,\com^*_0,\c^*_0,\s^*_0) = 0$
                 \hfill\comment{$\G_1-\G_3 $}
                \item \pcthen $\BAD_1 \leftarrow \true;\abort$
                 \hfill\comment{$\G_1-\G_3$}
                 \item \pcif $\IDAltV(\pk_b,\com^*_b,\c^*_b,\s^*_b) = 0$
                 \hfill\comment{$\G_2-\G_3 $}
                 \item \pcthen $\BAD_2 \leftarrow \true;$
                  \hfill\comment{$\G_2-\G_2 $}
                 \item \quad $\abort$
                  \hfill\comment{$\G_3 $}
            	\item \pcif $\c_0^*+\c_1^* \neq \Hash(\pk,\com^*_0,\com^*_1) $
		\item \quad \pcthen \pcreturn $0$
		\item \pcif $\IDV(\pk_0,\com_0,\c_0,\s_0) = 0 \lor \IDV(\pk_1,\com_1,\c_1,\s_1) =0$
		\item \quad \pcthen \pcreturn $0$
		 \item $\pcif m^* \notin \{m_1,..,m_q\} $
		 \item \quad \pcthen \pcreturn $0$
		\item \pcelse \pcreturn $1$
                

            \end{nicodemus}
        \end{minipage}
                       \begin{minipage}[t]{0.8\textwidth}
            \underline{$\G_4$}
            \begin{nicodemus}
            \item $b \sample \{0,1\}$
            \item $\BAD_2 \leftarrow \true$

            \item $(\pk_0,\sk_0)\sample\IDGen$
            \item $(\pk_1,\sk_1)\sample\IDGen$
            \item $pk \coloneqq (\pk_0,\pk_1)$
            \item $\sk \coloneqq (\sk_0, \sk_1)$
            \item \pcfor $i \in [q]$
                \item \quad $ \m_i \sample \mathsf{M}$
            	\item	\quad $(\com_{0,i},\st_{0,i})\sample\IDP_1(\sk_0,\m_i)$
		\item \quad $(\com_{1,i},\st_{1,i})\sample\IDP_1(\sk_1,\m_i)$ 
		\item \quad $ k_i = \Hash(\pk,\com_{0,i},\com_{1,i}) $
  		\item \quad $ e_i \sample \ChSet$
		\item \quad $\c_{b,i} = e_i$  
		\item \quad $ \c_{1-b,i} = k_i \oplus e_i $
		\item \quad $\s_{0,i} \sample\IDP_2(\sk_0,\st_{0,i},\c_{0,i})$
		\item \quad $ \s_{1,i} \sample\IDP_2(\sk_1,\st_{1,i},\c_{1,i})$
		\item \quad $\sigma_i \coloneqq (\com_{0,i},\c_{0,i},\s_{0,i}, \com_{1,i}, \c_{1,i},\s_{1,i})$
                \item \quad $\Q \leftarrow \Q \cup (m_i,\sigma_i) $
                \item $(\m*, \sigma^*)\gets\adv(\pk,\Q)$
                \item \pcparse $\sigma^* = (\com_0^*,\c_0^*,\s_0^*, \com_1^*, \c_1^*,\s_1^*)$
                \item \pcif $\IDAltV(\pk_0,\com^*_0,\c^*_0,\s^*_0) = 0 \land \IDAltV(\pk_0,\com^*_0,\c^*_0,\s^*_0) = 0$
                \item \pcthen $\BAD_1 \leftarrow \true;\abort$
                 \item \pcif $\IDAltV(\pk_b,\com^*_b,\c^*_b,\s^*_b) = 0$
                 \item \pcthen $\BAD_2 \leftarrow \true;$
                 \item \quad $\abort$
            	\item \pcif $\c_0^*+\c_1^* \neq \Hash(\pk,\com^*_0,\com^*_1) $
		\item \quad \pcthen \pcreturn $0$
		\item \pcif $\IDV(\pk_0,\com_0,\c_0,\s_0) = 0 \lor \IDV(\pk_1,\com_1,\c_1,\s_1) =0$
		\item \quad \pcthen \pcreturn $0$
		 \item $\pcif m^* \notin \{m_1,..,m_q\} $
		 \item \quad \pcthen \pcreturn $0$
		\item \pcelse \pcreturn $1$
                

            \end{nicodemus}
        \end{minipage}
    }}
    \caption{}
    \label{fig:x-soundess}
\end{figure}
\begin{theorem}
Let  $\ID$ be a unique identification scheme and $\Hash$ be a $(t'',\epsilon'')$ correlation intractable hash function. Suppose there exists a $(t,q,\epsilon)$-forger $\mathcal{F}$ breaking the security of $\Sig_{\ID,\Hash}$ against the existential forgery under the random message attack. Then there exists an adversary that $(t',q,\epsilon')$-breaks the $\imp^{\IDAltV}$ security of $\ID$ with $t' \approx t$ and
$$ \epsilon' \geq \frac{1}{2}(\epsilon + \epsilon'')$$
\end{theorem}
\begin{proof}

We define the event of Game $\G_i$ winning (returning 1) as $X_i$. Let $(\m_i,\sigma_i)$ denote the $i$-th random message and its signature. Let $(\m^*,\sigma^*)$ be the forgery output by $\mathcal{F}$.
\begin{description}[wide,itemindent=\labelsep]
\item [] \textbf{Game 0.} We define Game 0 as the existential unforgeability experiment with forger $\mathcal{F}$ on the signature scheme$\Sig_{\ID,\Hash}$ as shown in Figure 5. By definition, we have
$$\Pr[X_0] = \epsilon.$$
\item [] \textbf{Game 1.} In $\G_1$ we check if the signature is partially valid or not and set $\BAD_1$ to $\true$ and $\abort$ if it isn't. Which according to Lemma 1 and $\Hash$ being $(t'',\epsilon'')$ correlation intractable happens with at most $\epsilon''$ probability and so we have 
$$\Pr[X_1] = \Pr[X_0 \land \neg\BAD_1]  \geq \Pr[X_0] + \epsilon''.$$

\item [] \textbf{Game 2.} In $\G_2$ we pick a random bit $b$ in the beginning of the game and after getting the forged signature $\sigma^*$ which we parse as 
$$\sigma^* = (\com_0^*,\c_0^*,\s_0^*, \com_1^*, \c_1^*,\s_1^*),$$
we check whether $\IDAltV(\pk_b,\com^*_b,\c^*_b,\s^*_b)$ is zero and set the tag $\BAD_2$ to $\true$ if it is. Since this change is only internal to the game 
$$\Pr[X_1] = \Pr[X_2].$$
\item [] \textbf{Game 3.} In $\G_3$ we abort if $\BAD_2$ that we defined in the last game is set to $\true$. Since the game would have already aborted if the forged signature was not partially valid signature and $b$ was chosen randomly in the beginning, we have
$\Pr[\BAD_2] \leq \frac{1}{2}$, which implies
$$\Pr[X_3] = \Pr[X_2 \land \neg \BAD_2] \geq \frac{1}{2} \Pr[X_2].$$

\item [] \textbf{Game 4.} 
Game $\G_4$ is exactly like $\G_3$ except instead of always choosing $\c_{0,i}$ randomly from the $\ChSet$ and then calculating $\c_{1,i}$ accordingly, we choose $\c_{b,i}$ first and then calculate 
$\c_{1-b,i}$. Since the distribution of $(\c_{0,i},\c{1,i})$ does not change we have
$$\Pr[X_4] = \Pr[X_3].$$
We point out that in this game we can choose $(\m_i, \com_{b,i}, \c_{i,b}, \s_{b,i})$ first and then calculate 
$(\com_{1-b,i}, \c_{1-b,i}, \s_{1-b,i})$ and thus the signature $\sigma_i$ accordingly.

\end{description}

Now adversary $\adv$ simulates game $\G_4$. The $\adv$ receives $\pk_{b}$ and $(\tau_i, \com_{b,i},\c_{b,i},\s_{b,i})$ from the alternative impersonation game and proceeds to
run $\IDGen$ to obtain $\pk_{1-b}$ and
 calculate signatures on message $m_i := \tau_i$. As pointed out before it is possible to calculate $\sigma_i$ according to $(\tau_i, \com_{b,i},\c_{b,i},\s_{b,i})$.
\\
It remains to show how $\adv$ can break the alternative impersonation from the forged signature
$\sigma^* = (\com^*_{0},\c^*_{0},\s^*_{0}, \com^*_{1}, \c^*_{1},\s^*_{1})$ on message $m^*$ output by $\mathcal{F}$. We know that $\IDAltV(\sk_b,m^*,\com^*_b,\c^*_b,\s^*_b) =1$ (by game 2).  So $\adv$ can win the alternative impersonation game by outputting $(\m^*,\com^*_b,\c^*_b,\s^*_b)$.
\\
So putting all of this together we have 
$$\Pr[X_4] = \epsilon' \geq \frac{1}{2}(\epsilon + \epsilon'')$$
\end{proof}

\begin{lemma}
Let $\mathcal{F}$ be a forger that $(t,q,\epsilon)$-breaks the RMA security of the signature such that the forged signature it outputs is not partially valid. Then there exists adversary $\mathcal{A}$ that $(t'',\epsilon'')$-breaks the correlation intractability of the hash function $\Hash$ with 
$t \approx t''$ and 
$$\epsilon'' \geq \epsilon.$$
\end{lemma}

\begin{proof}
The correlation intractability adversary $\adv$ runs the unforgeability experiment by
running $\IDGen$ twice and obtaining two pairs of keys we name $(\sk_0,\pk_0)$ and $(\sk_1, \pk_1)$. The adversary now return 
$\pk := (\pk_0, \pk_1)$ to $\mathcal{F}$ as the public key and also
chooses random messages $m_1,...,m_q$ and signs them with the secret key 
$\sk := (\sk_0, \sk_1)$ to obtain the signatures $\sigma_1,...,\sigma_q$ and returns the $(m_i,\sigma_i)$ pairs to $\mathcal{F}$.
\\
Eventually, $\mathcal{F}$ returns a message and signature pair 
 $(\m^*, \sigma^*)$, from which $\adv$ extracts the solution that breaks the hash intractability as follows.
 \\
 First $\adv$ parses $\sigma^*$
as
$(\com^*_{0},\c^*_{0},\s^*_{0}, \com^*_{1}, \c^*_{1},\s^*_{1})$.
We assume that the forged signature is valid and since it is not partially valid, due to the uniqueness of the identification scheme
 we can write
$$\c^*_0 = f(\pk_0,\com^*_0)$$
$$\c^*_1 = f(\pk_1,\com^*_1).$$
Since we have assumed the forged signature is valid 
$$\Hash(\pk,\com^*_0,\com^*_1)=\c^*_0+\c^*_1 =  f(\pk_0,\com^*_0)+ f(\pk_1,\com^*_1) = g(\pk,\com^*_0,\com^*_1)$$
must hold.
So $\adv$ can $(t,\epsilon')$ break the $g$-correlation intractability of $\Hash$ where $g$ is defined as 
$$g(\pk=(\pk_0,\pk_1), \com_0, \com_1) =  f(\pk_0,\com_0)+ f(\pk_1,\com_1).$$
Adversary $\adv$ succeeds at giving a solution that breaks the correlation intractability of $\Hash$ whenever $\mathcal{F}$ succeeds at forging a valid signature so
$$\epsilon''\geq\epsilon.$$

\end{proof}



