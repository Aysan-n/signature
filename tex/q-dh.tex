\section{Instantiation of dual tag ID from q-DH}
We describe the identification scheme as in figure 7.
Throughout this section we will write $D(\tau)$ shorthand for 
$\phfEval(\kappa,\tau)$ and $d(\tau)$ shorthand for the function computing
$(a,b) \leftarrow \phfTE(\eta,\tau)$ and returning $ax+b$.
\begin{figure}[htb!]
    \centering
    \nicoresetlinenr
    \fbox{
        \begin{minipage}[t]{0.5\textwidth}
            \underline{$\IDGen(\pms)$:}
            \begin{nicodemus}
            	\item $g_1,g_2 \sample \G$
		\item $x \sample \mathbb{Z}_p$
                \item $X = g_2^x$
                \item $(\kappa, \eta) \sample \phfTG(g_2,X)$
                \item $\pk \coloneqq (g_1,g_2,\kappa)$
                \item $\sk \coloneqq (\pk,x,\eta)$
                \item $\ChSet \coloneqq \mathbb{Z}_p$
                \item $\pcreturn (\sk,\pk)$
            \end{nicodemus}
            
            \underline{$\IDV(\pk,\tau,\com,\c,\s)$}:
            \begin{nicodemus}
            	\item $\pcparse \com = (\hat{g},R,\hat{R})$
      		\item \pcif $R = g_2^\s \cdot D(\tau)^{-\c} \land$
		\item$ \quad \hat{R} = \hat{g}\cdot g_1^{-\c}$
                 \item \quad \pcthen \pcreturn 1
                 \item \pcelse \pcreturn 0
            \end{nicodemus}
            

            
        \end{minipage}
        \begin{minipage}[t]{0.5\textwidth}
            \underline{$\IDP_1(\sk,\tau):$}
            \begin{nicodemus}
            	\item $r \sample \mathbb{Z}_p$
                \item $\st \coloneqq (\tau,r )$
                \item $\hat{g} \coloneqq g_1^{\frac{1}{d(\tau)}}$
                \item $R \coloneqq g_2^r$
                \item $\hat{R} \coloneqq \hat{g}^r$
                \item $\com := (\hat{g},R,\hat{R})$
                \item $\pcreturn (\com,\st)$
            \end{nicodemus}
            
              \underline{$\IDP_2(\sk,\st,\c):$}
            \begin{nicodemus}
            	\item $\pcparse \st = (\tau,r)$
		\item $\pcparse \sk = x$
                \item $\pcreturn \s =\c\cdot d(\tau) + r \mod p$
            \end{nicodemus}
                        \underline{$\IDAltV(\sk,\tau,\com, \c,\s)$}:
            \begin{nicodemus}
                \item \pcparse $\com = (\hat{g},R,\hat{R})$
                \item \pcparse $\sk = (\pk,x,\eta)$
                \item \pcif $\hat{g} = g_2^{\frac{1}{d(\tau)}}$ 
                \item \quad \pcthen \pcreturn 1
                \item \pcelse \pcreturn 0
            \end{nicodemus}
        \end{minipage}
    }
    \caption{Instantiation 1}
    \label{fig:prf-security}
\end{figure}

\begin{definition}[$q$-DH Assumption]
We say an adversary $\adv$ breaks the $q$- Diffie Hellman ($q$-DH) assumption if it's running time is bounded by $t$ and
$$\Pr[g^{\frac{1}{x}} \sample \adv(g,g^x,...,g^{x^q})] \geq \epsilon,$$
where 
$g \sample \G$ and $x \sample \mathbb{Z}_p^*$.
We require that the $q$-DH assumption holds meaning that no adversary can $(t, \epsilon)$ break the $q$-DH problem for a polynomial $t$ and a non-negligible $\epsilon$.
\end{definition}
\begin{lemma}
	Let $\phf$ be a $(m,n, \lambda, \delta)$-programmable hash function and 
	$$(\kappa, \eta)\sample \phfGen(g,X)$$
	and $\tau_1,...,\tau_q \in \mathbb{Z}_p$ be such that
	$(k_i,l_i) \sample \phfEval(\eta,\tau_i)$.
	We define the polynomial $f$ as 
	$$f(Y) \coloneqq \prod_{i=1}^q (k_i + l_iY).$$
	Given $\{g^{x^i}\}_{i=0,...,q}$, let us define $g_1 = g^{\theta f(x)}$. For any $\tau_i$ where $i \in [0,q]$ it is easy to compute $g_1^\frac{1}{d(\tau_i) }$. Furthurmore if $k_i \neq 0$ for all $i \in [0,q]$, given 
	$g_1^\frac{1}{d(\tau^*) }$ where $\tau \notin \{\tau_1,...,\tau_q\}$ and $(k^*,l^*) \sample \phfEval(\kappa, \tau^*)$ one can easily compute $g^\frac{1}{d(\tau^*)}$.
	
\end{lemma}

\begin{proof}
	To compute $g_1^{\frac{1}{d(\tau_i)}}$ for $i \in [1,q]$, let $f_i$ be defined as
	$$f_i(Y) = \frac{f(Y)}{k_i+l_iY} = \prod_{j=1, j \neq i}^q (k_i + l_iY).$$
	We can write $f_i$ as 
	$f_i(Y) = \sum_{j=0}^{q-1} \beta_jY^j$ while calculating its coefficient.  Now if we denote $g^{x^j}$ by $d_j$ we can calculate
	$$\hat{g}_i = \prod_{j=0}^{q-1} d_j^{\theta\beta_j}$$
	hence
	$$\hat{g}_i = g^{\theta f_i(x)} = g_1^{\frac{ f_i(x)}{f(x)}}=   g_1^{\frac{1}{d(\tau_i)}}.$$
	Now to compute $g^\frac{1}{d(\tau^*)}$ by long division we can write $f(Y)$ as 
	$$f(Y) = (k^*+l^*Y)\alpha(Y) + \beta$$
	where the coefficients of $\alpha(Y) = \sum_{i=0}^{q-1} \alpha_iY^i$ are easily computable. So we can write $\frac{f(Y)}{k^*+l^*Y}$ as $\alpha(Y) + \frac{\beta}{k^*+l^*Y}$ and so we have
	$$\hat{g} = g_1^{\frac{1}{d(\tau^*)}} =g^{\frac{\theta f(x)}{d(\tau^*)}}=g^{\frac{\theta f(x)}{k^* + l^*x}} = g^{\theta\cdot(\alpha(x) + \frac{\beta}{k^*+l^*x})} $$
	Since $k_i +l_iY$ divides $f(Y)$ for all $i\in [1,q]$ and $f(Y)$ has a degree of $q$ and $k_i \neq 0$, $l^*Y$ does not divide $f(Y)$ and thus $\beta$ is non zero and we can compute
	$$w \leftarrow \Big{(} \hat{g}^{\frac{1}{\theta}} \cdot \prod_{i=0}^{q-1} d_i^{-\alpha_i}\Big{)}^{\frac{1}{\beta}}$$
	Hence, we have computed
	$$w  = g^{\frac{1}{k^*+l^*x}} = g^\frac{1}{d(\tau^*)}.$$
	
\end{proof}


\begin{theorem}
Suppose that there exists a $(t, q, \epsilon)$-forger $\fdv$ breaking the $\imp^{\IDAltV}$ of the $\ID_{q-\text{DH}}$ identification scheme. Then there exists an adversary $\adv$ that $(t',q+1,\epsilon')-$ breaks the $q+1$-DH assumption with $t \approx t'$ and 
$\epsilon' \geq \epsilon.$
\end{theorem}
\begin{proof}
We define the event of Game $\G_i$ winning (returning 1) as $X_i$. Let $\tau_i$ denote the $i$-th random tag in the alternative impersonation game. Let $(\tau^*,\com^*, \c^*, \s^*)$ be the forgery output by $\mathcal{F}$.
\begin{description}[wide,itemindent=\labelsep]
\item [] \textbf{Game 0.} We define Game 0 as the alternative impersonation game and so by definition
$$\Pr[X_0] = \epsilon.$$
\item [] \textbf{Game 1.} In this game we choose the tags $\tau_1,...,\tau_q$ all in the beginning. This does not effect the success probability of the adversary so 
$$\Pr[X_1] = \Pr[X_0].$$
\item [] \textbf{Game 2.} In this game we compute
$(l_i,k_i) \leftarrow \phfEval(\kappa,\tau_i)$ for every $\tau_i$ and set $\BAD$ to $\true$ if for any $i$, $l_i$ is zero. We also compute $(l^*,k^*) \leftarrow \phfEval(\kappa,\tau^*)$ when the adversary has output the forgery and set $\BAD$ to $\true$ if $l^*$ is not zero.
\\
By the $(1, \poly, \gamma, \delta)-$programmability of $D$ we have
$$\Pr[\neg \BAD] \geq \delta$$
\item [] \textbf{Game 3.} This game is exactly like the last game except that we abort the game if $\BAD$ is $\true$ which means
$$\Pr[X_3] = \Pr[X_2 \land \neg \abort] \geq \delta \cdot \Pr[X_2]$$
\end{description}


\begin{figure}[htb!]
    \centering
    \nicoresetlinenr
    \fbox{
        \begin{minipage}[t]{0.7\textwidth}
            \underline{$\G_0-\G_3$}
            \begin{nicodemus}
            	\item $\BAD \leftarrow \false$
	\hfill\comment{$\G_2 - \G_3$}
           	 \item \pcfor $i \in [q]$
	 \hfill\comment{$\G_1 - \G_3$}
                \item \quad $ \tau_i \sample \mathbb{Z}_p$
                \hfill\comment{$\G_1 - \G_3$}
           \item $g_1,g_2 \sample \G$
		\item $x \sample \mathbb{Z}_p$
                \item $X = g_2^x$
                \item $(\kappa, \eta) \sample \phfTG(g_2,X)$
                \item $\pk \coloneqq (g_1,g_2,\kappa)$
                \item $\sk \coloneqq (\pk,x,\eta)$

		\item $\Q \leftarrow \emptyset$
                \item \pcfor $i \in [q]$
                \item \quad $ \tau_i \sample \mathbb{Z}_p$
                \hfill\comment{$\G_0$}
                \item  \quad $r_i \sample \mathbb{Z}_p$
                \item  \quad $\st_i \coloneqq (\tau_i,r_i )$
                \item \quad $(l_i,k_i) \leftarrow \phfEval(\kappa,\tau_i)$
                \hfill\comment{$\G_2-\G_3$}
                \item \quad \pcif $l_i = 0$
                \hfill\comment{$\G_2-\G_3$}
                \item \quad \quad \pcthen $\BAD \leftarrow \true$
                \hfill\comment{$\G_2-\G_3$}
                \item \quad  $\hat{g_i} \coloneqq g_1^{\frac{1}{d(\tau_i)}}$
                \item \quad  $R_i \coloneqq g_2^{r_i}$
                \item  \quad $\hat{R}_i \coloneqq \hat{g}^{r_i}$
                \item  \quad $\com_i := (\hat{g_i},R_i,\hat{R_i})$
                \item \quad $ \c_i \sample \mathbb{Z}_p$
                \item \quad $\s_i = \c_i\cdot d(\tau_i) + r_i$
                \item \quad $\Q \leftarrow \Q \cup (\tau_i,\com_i,\c_i,\s_i) $
                \item $(\tau^*, \com^*,\c^*,\s^*)\gets\adv(\pk,\Q)$
                 \item \quad $(l^*,k^*) \leftarrow \phfEval(\kappa,\tau^*)$
                \hfill\comment{$\G_2-\G_3$}
                \item \quad \pcif $l^* \neq 0$
                \hfill\comment{$\G_2-\G_3$}
                \item \quad \quad \pcthen $\BAD \leftarrow \true$
                \hfill\comment{$\G_2-\G_3$}
        
                \item \quad \pcif $\BAD$
                \hfill\comment{$\G_3$}
                \item \quad \quad \pcthen $\abort$
                \hfill\comment{$\G_3$}

                \item $\pcif \tau^* \notin \{\tau_1,...,\tau_q\} \land \IDAltV(\sk, \tau^*, \com^*,\c^*,\s^*) = 1$
                \item $\pcthen \pcreturn 1$
                \item $\pcelse \pcreturn 0$

            
            \end{nicodemus}
        \end{minipage}
    }
    \caption{}
    \label{fig:x-soundess}
\end{figure}

The $q$-DH adversary $\adv$ receives $d_0,...,d_q$ as inputs where $d_i = g_2^{x^i}$ and simulates the soundness experiment as follows
\subsection*{Key Generation:}
Adversary $\adv$ first chooses random $\tau_1,...,\tau_q$ from $\mathbb{Z}_p$. 
\subsection*{Key Generation:} The adversary $\adv$ first chooses random $\tau_1,...,\tau_q$ from $\mathbb{Z}_p$. 
Now $\adv$ has $g_2$ as $d_0$ and sets 
$$X \coloneqq d_1 = g_2^x.$$
Now $\adv$ runs run the following 
$$(\kappa, \eta)\sample \phfGen(g_2,X)$$
just like the original key generation algorithm. Then using $\eta$, $\adv$ runs 
$$(k_i,l_i) \sample \phfEval(\eta,\tau_i)$$
for every $i \in [1,q]$.
\\
$\adv$ computes $g_1$ as in Lemma 3, such that
$$g_1 = g_2^{\theta f(x)}.$$
\\
$\adv$ can also calculates 
$X = g_1^x = g_2^{xf(x)}$ similarly since $Yf(Y)$ has a degree equal to $q+1$.
\\
Adversary $\adv$ returns $(g_1,g_2,\kappa)$ as the public key to $\fdv$. This is indistinguishable from the normal key generation for $\mathbb{F}$ since $g_1$ is randomly distributed in $\G$.

\subsection*{Transcript Generation:}
Now adversary $\adv$ has compute $(\com_i,\c_i,\s_i)$ for all $\tau_i$ where $i \in [1,q]$.
\\
According to Lemma 2 $\adv$ can compute $\hat{g}_i = g_1^{\frac{1}{d(\tau_i)}}$.
\\
To complete the computation $\adv$ chooses  $\c_i,\s_i\sample\mathbb{Z}_p$ and computes
$$R = g_2^{\s_i} \cdot D(\tau)^{-\c_i}$$
$$ \hat{R} = \hat{g_i}\cdot g_1^{-\c_i}.$$
Now $\adv$ returns $(\com_i = (\hat{g_i},R,\hat{R}),\c_i,\s_i)$ to $\fdv$ and this is indistinguishable from the normal transcript generation for $\fdv$ since if we define $r$ to be $r = \s-\c x$ then $R = g_1^x$ and 
$\hat{R} = \hat{g}^x$ and also since $\s$ and $\c$ are uniformly distributed in $\mathbb{Z}_p$ so is $r$.
\end{proof}

\subsection*{Breaking the $q+1$-DH:}
Eventually forger $\fdv$ returns a forgery $(\tau^*,\com^*,\c^*,\s^*)$ we assume that $\fdv$ wins the game and thus $\tau^* \notin \{\tau_1,...,\tau_q\}$ and 
$\IDAltV(\sk, \tau^*,\com^*,\c^*,\s^*) = 1$ which means if we parse $\com^*$ as $(\hat{g}^*,R^*,\hat{R}^*)$
$$\hat{g}^* = g_1^{\frac{1}{d(\tau^*)}} $$
will hold.
\\
According to Lemma 3, $\adv$ can now compute 
$$g^\frac{1}{d(\tau^*)} = g^\frac{1}{k^*+l^*x}  \stackrel{(*)}{=} g^\frac{1}{l^*x}$$
where
$$w:=(k^*,l^*) \sample \phfEval(\kappa, \tau^*)$$
and $(*)$ uses that $k^* = 0$ by Game 3.
\\
Ultimately, $\adv$ can compute
$w^{l^*} = g^\frac{1}{x}$ 
and return it as the solution to the $q+1-$DH problem.

