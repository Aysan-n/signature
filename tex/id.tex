\section{Identification Scheme}

\begin{definition}[Canonical Tag-based Identification Scheme]
A canonical tag-based  identification (tag-ID) scheme is defined as the probabilistic algotithms $\ID \coloneqq(\IDGen, \IDP, \IDV)$ where
\begin{itemize}
\item$\IDGen$ returns a public key and secret key $(\pk,\sk)$. We assume that $\pk$ defines the challenge set $\ChSet$ and tag space $\tgSet$.

\item The prover algorithm $\IDP = (\IDP_1,\IDP_2)$ is split into two algorithms. $\IDP_1$ takes the secret key $\sk$ and a tag $\tau$ from the tag space $\mathsf{M}$ as the input and returns a commitment $\com$ and a state $\st$. $\IDP_2$ takes the secret key $\sk$, the state $\st$ and a challenge $\c$ as an input and returns a response $\s$. 

\item The deterministic verifier algorithm $\IDV$ takes the public key $\pk$, the tag $\tau$, the commitment $\com$, the challenge $\c$ and the response $\s$ as an input and outputs a decision, 1 (acceptance) or 0 (rejection).
\end{itemize}
For correctness we require that for all $k \in \mathbb{N}$,
$(\pk,\sk) \in \IDGen(1^k)$,
all
$(\com,\st) \in \IDP_1(\sk, \tau)$,
all $\c \in \ChSet$ and all 
$\s \in \IDP_2(\sk,\st,\c)$,
we have
$$ \IDV(\pk,\com,\c,\s) = 1.$$
\begin{figure}[H]
  \centering
   \nicoresetlinenr
\vspace{2em}
 \fbox{
 \pseudocode{%
 \textbf{ Prover($\sk,\tau$)} \< \< \textbf{ Verifier($\pk$)}  \\[0.1\baselineskip][\hline]
  \<\< \\[-0.5\baselineskip]
(\com,\st) \sample \IDP_1(\sk,\tau)
  \< \sendmessageright*{\tau,\com} \< \\[-2ex]
  \<\< \c \sample \ChSet \\[-4ex]
  \< \sendmessageleft*{\c} \< \\
  \s \sample\IDP_2(\sk,\st,\c) 
  \< \sendmessageright*{\s} \< \\[-2ex]
   \<\< d \coloneqq \IDV(\pk,\com,\c,\s)
  }
  
 }
  \caption{Canonical Tag-based Identification Scheme}
 \end{figure}

\end{definition}



\begin{definition}[Dual Tag-ID]
A dual canonical tag based identification scheme (dual tag-ID) is a identification scheme $\ID$, with an additional algorithm $\IDAltV$ called the alternative verification algorithm that takes the secret key $\sk$, the tag $\tau$, the commitment $\com$, the challenge $\c$ and the response $\s$ as an input and outputs a decision, 1 (acceptance) or 0 (rejection).
\\
For the correctness of this scheme in additional to the correctness defined before we require that for all $k \in \mathbb{N}$,
$(\pk,\sk) \in \IDGen(1^k)$,
all
$(\com,\st) \in \IDP_1(\sk, \tau)$,
all $\c \in \ChSet$ and all 
$\s \in \IDP_2(\sk,\st,\c)$,
we have
$$ \IDAltV(\sk,\tau,\com,\c,\s) = 1.$$
\end{definition}

\begin{definition}[Existential Unforgeablilty against Passive Attacks]
A dual tag-ID scheme is said to be $(t,q,\epsilon)\hyph \ufpa$ secure, if for all adversary $\adv$ running in time at most $t$ we have
$$\Pr[q\hyph\xsoundness_{\ID}(\adv)]=1] \leq \epsilon.$$
\end{definition}

Unlike most commonly used identification schemes the canonical tag based ID schemes we use are not Honest Verifier Zero Knowledge (HVZK) and instead have some different soundness property.

\begin{figure}[htb!]
    \centering
    \nicoresetlinenr
    \fbox{
        \begin{minipage}[t]{0.8\textwidth}
            \underline{$\textbf{Game } q\hyph\xsoundness_{\ID}(\adv)$}
            \begin{nicodemus}
            	\item $(\sk,\pk)\sample\IDGen$
		\item $\Q \leftarrow \emptyset$
                \item \pcfor $i \in [q]$
                \item \quad $ \tau_i \sample \mathsf{M}$
                \item \quad $(\com_i,\st_i)\sample\IDP_1(\sk,\tau_i)$
                \item \quad $ \c_i \sample \ChSet $
                \item \quad $\s_i \sample\IDP_2(\sk,\st_i,\c_i)$
                \item \quad $\Q \leftarrow \Q \cup (\tau_i,\com_i,\c_i,\s_i) $
                \item $(\tau^*, \com^*,\c^*,\s^*)\gets\adv(\pk,\Q)$
                \item $\pcif \tau^* \notin \{\tau_1,...,\tau_q\} \land \IDAltV(\sk, \tau^*, \com^*,\c^*,\s^*) = 1$
                \item $\pcthen \pcreturn 1$
                \item $\pcelse \pcreturn 0$
            \end{nicodemus}
        \end{minipage}
    }
    \caption{}
    \label{fig:x-soundess}
\end{figure}




\begin{definition}[$\uniqueness$]
We say the identification scheme $\ID \coloneqq(\IDGen, \IDP ,\ChSet, \IDV)$ is
unique if for every $(\sk,\pk)\in\IDGen$ and every $(\com,\st)\in\IDP_1(\sk,\tau)$,
$$\big{|}\{\c \in \ChSet \mid \exists \s : \IDV(\pk,\com,\c,\s) = 1 \land \IDAltV(\sk, \com, \C,  \s) \neq 1 \}\big{|} = 1.$$
This means there exist a (not necessarily polynomial time) function we call the uniqueness function such as $f$ that
$$f(\pk,\com) = \c.$$
\end{definition}

